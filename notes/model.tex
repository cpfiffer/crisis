\documentclass{article}

\title{Crisis Learning}

\usepackage{amsmath}
\usepackage{palatino}
\usepackage{tikz}
\usetikzlibrary{positioning, calc, 
shapes.geometric, shapes, 
shapes.multipart, arrows.meta, 
arrows, decorations.markings, 
external, trees}

\begin{document}

\tikzstyle{Arrow} = [
	thick, 
	decoration={
		markings,
		mark=at position 1 with {
			\arrow[thick]{latex}
			}
		}, 
	shorten >= 3pt, preaction = {decorate}
	]

\maketitle

Howdy hey, gang.

\begin{figure}[p]
    \centering
\begin{tikzpicture}
    \node (1) {$S_0$};
    \node [right =of 1] (2) {$S_1$};
    \node [right =of 2] (3) {$S_2$};
    \node [right =of 3] (4) {$\dots$};
    \node [right =of 4] (5) {$S_t$};
    
    \draw[Arrow] (1.east) -- (2.west);
    \draw[Arrow] (2.east) -- (3.west);
    \draw[Arrow] (3.east) -- (4.west);
    \draw[Arrow] (4.east) -- (5.west);

    \node [below =of 2] (6) {$R_1$};
    \node [below =of 3] (7) {$R_2$};
    \node [below =of 5] (9) {$R_t$};
    
    \draw[Arrow] (2.south) -- (6.north);
    \draw[Arrow] (3.south) -- (7.north);
    \draw[Arrow] (5.south) -- (9.north);

    % \draw[Arrow] (2) to [out=25, in=160] (4); 
\end{tikzpicture}
\end{figure}

\end{document}