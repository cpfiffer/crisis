\documentclass{article}

\title{Crisis Learning}
\author{Cameron Pfiffer}

\usepackage{amsmath}
\usepackage{palatino}
\usepackage{tikz}
\usepackage{amssymb}
\usepackage{todonotes}
\usepackage{optidef}

\usetikzlibrary{positioning, calc, 
shapes.geometric, shapes, 
shapes.multipart, arrows.meta, 
arrows, decorations.markings, 
external, trees}

\begin{document}

% Expectation operators
\newcommand{\Var}{\text{Var}}

% TIKZ crap
\tikzstyle{Arrow} = [
	thick, 
	decoration={
		markings,
		mark=at position 1 with {
			\arrow[thick]{latex}
			}
		}, 
	shorten >= 3pt, preaction = {decorate}
	]

\maketitle

Howdy hey, gang. The goal of this document is to describe the rough shape of an economy with a simple state-based structure. 

\section{Model styles}

In all models, it is assumed that the initial state distribution is categorically distributed with probabilities $\Pi = [\pi_1, \pi_2, \dots, \pi_Q]$ for $Q$ states.

The state transition matrix is given by $A$. It is assumed that all know the transition matrix. 

\subsection{Simple Gaussian}

Figure \ref{fig:simple-gauss} demonstrates an extremely simple economy where a latent state variable $S_t$ evolves over time. In each possible state $S_t \in \mathbb{S}$, the location and scale parameters ($\mu(S_t)$ and $\Sigma(S_t)$) of the Gaussian payoff vector $f_t$ may vary:

\begin{enumerate}
    \item The degenerate constant mean/variance condition
    $\Sigma(S_t)= \Sigma$ and a changing mean payoff vector, $\mu(S_t) = \mu$.
    \item A constant variance term $\Sigma(S_t) = \Sigma(S_t') = \Sigma$ and a changing mean payoff vector, $\mu(S_t) \ne \mu(S_t')$.
    \item A varying variance term $\Sigma(S_t) \ne \Sigma(S_t')$ and a constant mean payoff vector, $\mu(S_t) = \mu$.
    \item The "everything changes" case where both $\mu(S_t)$ and $\Sigma(S_t)$ vary.
\end{enumerate}

Regardless of the choice of method above, going forward I will simply denote means and variances as $\mu$ and $\Sigma$ to reduce notational costs, but keep in mind that they are implicit functions of macroeconomic state $S_t$.

\begin{figure}\label{fig:simple-gauss}
    \centering
\begin{tikzpicture}
    \node [circle,draw=black,fill=lightgray](1) {$S_0$};
    \node [circle,draw=black,fill=lightgray,right = of 1] (2) {$S_1$};
    \node [circle,draw=black,fill=lightgray,right = of 2] (3) {$S_2$};
    \node [right =of 3] (4) {$\dots$};
    \node [circle,draw=black,fill=lightgray,right = of 4] (5) {$S_t$};
    
    \draw[Arrow] (1.east) -- (2.west);
    \draw[Arrow] (2.east) -- (3.west);
    \draw[Arrow] (3.east) -- (4.west);
    \draw[Arrow] (4.east) -- (5.west);

    % \node [circle,draw=black,fill=lightgray,below =of 2] (6) {$\Sigma(S_1)$};
    % \node [circle,draw=black,fill=lightgray,below =of 3] (7) {$\Sigma(S_2)$};
    % \node [circle,draw=black,fill=lightgray,below =of 5] (8) {$\Sigma(S_t)$};

    % \node [circle,draw=black,fill=lightgray,below =of 6] (9) {$\mu(S_1)$};
    % \node [circle,draw=black,fill=lightgray,below =of 7] (10) {$\mu(S_2)$};
    % \node [circle,draw=black,fill=lightgray,below =of 8] (11) {$\mu(S_t)$};
    
    \node [circle,draw=black,below =of 2] (12) {$f_1$};
    \node [circle,draw=black,below =of 3] (13) {$f_2$};
    \node [circle,draw=black,below =of 5] (14) {$f_t$};

    % \node [circle,fill=lightgray,draw=black,below right = 1cm and 0.05cm of 2] (15) {$\Delta c_1$};
    % \node [circle,fill=lightgray,draw=black,below right = 1cm and 0.05cm of 3] (16) {$\Delta c_2$};
    % \node [circle,fill=lightgray,draw=black,below right = 1cm and 0.05cm of 5] (17) {$\Delta c_t$};
    % \node [circle,draw=black,below =of 2] (10) {$f_1$};
    % \node [circle,draw=black,below =of 3] (11) {$f_2$};
    % \node [circle,draw=black,below =of 5] (12) {$f_t$};
    
    \draw[Arrow] (2.south) -- (12.north);
    \draw[Arrow] (3.south) -- (13.north);
    \draw[Arrow] (5.south) -- (14.north);

    % \draw[Arrow] (2.south) -- (15.north);
    % \draw[Arrow] (3.south) -- (16.north);
    % \draw[Arrow] (5.south) -- (17.north);

    % \draw[Arrow] (6.south) -- (9.north);
    % \draw[Arrow] (7.south) -- (10.north);
    % \draw[Arrow] (8.south) -- (11.north);

    % \draw[Arrow] (9.south) -- (12.north);
    % \draw[Arrow] (10.south) -- (13.north);
    % \draw[Arrow] (11.south) -- (14.north);

    % \draw[Arrow] (2) to [out=25, in=160] (4); 
\end{tikzpicture}
\caption{
    Depiction of the underlying economic process. $S_t$ is a Markov state process drawn from $S_t \sim P(S_t, S_{t-1})$, while $f_t$ is a multivariate Gaussian of dimension $N$ (one for each firm). Payoffs $f_t$ are Gaussian only conditional on $S_t$, i.e. $f_t \mid S_t \sim \mathcal{N}(\mu(S_t), \Sigma(S_t))$.
}
\end{figure}

Investors do not observe all elements of $f_t$ simultaneously. Rather, they observe them sequentially --- denote a partition of $f_t$ after $n \le N$ firms have been observed with $f_{t,1:n}$. Write the distributions of observed payoffs $f_{t,A}$ and the payoffs yet to be observed $f_{t,B}$ as the partitions

\begin{align*}
    f_t &= \begin{bmatrix}
        f^A_t & f^B_t
    \end{bmatrix}' \\ 
    f^A_{t} &\sim \mathcal{N}(\mu^A, \Sigma^A) \\
    f^B_{t} &\sim \mathcal{N}(\mu^B, \Sigma^B) \\
\end{align*}

\noindent where 

$$
E[f_t \mid S_t] = \mu = \begin{bmatrix}
    \mu_A \\
    \mu_B
\end{bmatrix}
$$

The superscripts $A$ and $B$ are used to indicate a subsetting operation where $\mu^A = \mu_{1:n}$ and $\mu_B = \mu_{n+1:N}$. The covariance matrices can be similarly partitioned into a block matrix

$$
\Var[f_t \mid S_t] = \underset{N \times N}{\Sigma} = \begin{bmatrix}
    \underset{n\times n}{\Sigma^A} & \underset{n\times N-n}{\Sigma^{AB}} \\
    \underset{N-n\times n}{\Sigma^{BA}} & \underset{N-n\times N-n}{\Sigma^B} \\
\end{bmatrix}
$$

Note, however, that $f_t^A$ is observed, and is no longer a stochastic variable. However, it was \textit{drawn} from a distribution correlated to $f_t^B$, and thus can be used as conditioning information to more precisely determine the distribution of the payoffs to be revealed, $f_t^B$. It can be shown \todo{Actually show this?} that, conditional on observing $f_t^A$, the distribution of $f_t^B$ is

$$
f_t^B \mid f_t^A, S_t \sim \mathcal{N} (\overline \mu^B, \overline \Sigma^B)
$$

\noindent for conditional parameters

\begin{align*}
    \overline \mu^B &= \mu^B + \Sigma^{BA} (\Sigma^{A})^{-1}(f_t^A - \mu^A)\\
    \overline \Sigma^B &= \Sigma^B + \Sigma^{BA} (\Sigma^{A})^{-1}\Sigma^{AB}
\end{align*}

\newcommand{\Shape}{\mathbf{\Omega}}

I assume that, for any state, the covariance matrix of firm payoffs is drawn from an inverse Wishart distribution parameterized by the shape matrix $\Shape$ and precision $\nu$. The distribution of $\Sigma$ holds even for cases where the covariance matrix does not change across states\footnote{In models where the covariance matrix changes with state (cases 3 and 4), the inverse Whishart distribution can still be used as the distribution collapses with certainty as $\nu \rightarrow \infty$.}. The matrix $\Shape$ determines the fundamental "shape" of the covariance structure, in that the mean of the distribution of $\Sigma$ is 

$$
E[\Sigma \mid S_t] = \frac{\Shape}{\nu - N - 1}
$$

The average \textit{covariance} can vary substantially in terms of scale as $\nu$ changes, but the average \textit{correlation} remains the same regardless of $\nu$. Any two draws $\Sigma_1$ and $\Sigma_2$ can have highly varied behavior. For example, $\Sigma_1$ might suggest a negative correlation in the payoffs of two firms, while $\Sigma_2$ could suggest a positive correlation. The inverse Wishart distribution is advantageous because the covariance matrix governing firm payoffs can vary meaningfully between states, and the distribution's properties are well-known.

\subsection*{The joint density}

\newcommand{\Slist}{S_0, \dots, S_t}
\newcommand{\Flist}{f_0, \dots, f_{t-1}, f_{t}^A}
\newcommand{\States}{\mathbf{S}}
\newcommand{\Payoffs}{\mathbf{f}}

The physical joint density of the economy is defined in terms of a particular set of states $\States = \Slist$ and a set of payoffs $\Payoffs = \Flist$. The chain rule of probability allows us to factor this probability as

$$
P(\States, \Payoffs, \Sigma, \mu) = P(\Payoffs \mid \States) P(\Sigma, \mu \mid \States) P(\States)
$$

The second part of the term above is a function of $A$ and $\Pi$. The first element of a particular path $\Slist$ is drawn using probabilities $\Pi$, and the underlying state transitions according to the entries in the matrix $A$. 

$$
P(\Slist) = P(\States) = P(S_0)P(S_1 \mid S_0) \dots P(S_t \mid S_{t-1})
$$

The term $P(S_0)$ is either $\pi_1$ or $\pi_2$, depending on $S_0$. Denote this as $\pi(S_0)$. Additionally, denote the transition probability from $S_{t-1}$ to $S_t$ as $A(S_{t-1}, S_t)$. The above equation can then be rewritten as

\begin{align*}
    P(\States) = \pi(S_0) \prod_{i=1}^t A(S_{i-1}, S_{i})
\end{align*}

The state space of $\States$ is $Q^t$. Each "path" of states $\States$ maps to an element on a discrete table of probabilities. Computing this table is computationally difficult but can be achieved with robust forward-backward passes.

\subsubsection*{Prices}

I assume for the moment that there is a representative agent investor with utility

\newcommand{\U}{\sum_{j=0}^\infty \rho^j u(C_{t+j})}
$$
U = \U
$$

\noindent for a utility function 

$$
u(C_t) = \frac{C_t^{1-\gamma}}{1 - \gamma}
$$

One way to model this economy is to allow wealth to differ in each time period, but this is generally a singificant amount of record keeping and seems to result in a particularly complex stochastic discount factor. 


\end{document}