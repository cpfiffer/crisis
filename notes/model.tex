\documentclass{article}

\title{Crisis Learning}
\author{Cameron Pfiffer}

\usepackage{amsmath}
\usepackage{palatino}
\usepackage{tikz}
\usepackage{amssymb}
\usepackage{todonotes}

\usetikzlibrary{positioning, calc, 
shapes.geometric, shapes, 
shapes.multipart, arrows.meta, 
arrows, decorations.markings, 
external, trees}

\begin{document}

% Expectation operators
\newcommand{\Var}{\text{Var}}

% TIKZ crap
\tikzstyle{Arrow} = [
	thick, 
	decoration={
		markings,
		mark=at position 1 with {
			\arrow[thick]{latex}
			}
		}, 
	shorten >= 3pt, preaction = {decorate}
	]

\maketitle

Howdy hey, gang. The goal of this document is to describe the rough shape of an economy with a simple state-based structure. 

\section{Model styles}

In all models, it is assumed that the initial state distribution is categorically distributed with probabilities $\Pi = [\pi_1, \pi_2]$. We'll keep it simple for now and use only two possible states, but in general everything should function with an arbitrary number of states. 

\subsection{Simple Gaussian}

Figure \ref{fig:simple-gauss} demonstrates an extremely simple economy where a latent state variable $S_t$ evolves over time. In each possible state $S_t \in \mathbb{S}$, the location and scale parameters ($\mu(S_t)$ and $\Sigma(S_t)$) of the Gaussian return vector $R_t$ may vary:

\begin{enumerate}
    \item The degenerate constant mean/variance condition
    $\Sigma(S_t)= \Sigma$ and a changing mean return vector, $\mu(S_t) = \mu$.
    \item A constant variance term $\Sigma(S_t) = \Sigma(S_t') = \Sigma$ and a changing mean return vector, $\mu(S_t) \ne \mu(S_t')$.
    \item A varying variance term $\Sigma(S_t) \ne \Sigma(S_t')$ and a constant mean return vector, $\mu(S_t) = \mu$.
    \item The "everything changes" case where both $\mu(S_t)$ and $\Sigma(S_t)$ vary.
\end{enumerate}

Regardless of the choice of method above, going forward I will simply denote means and variances as $\mu$ and $\Sigma$ to reduce notational costs, but keep in mind that they are implicit functions of macroeconomic state $S_t$.

\begin{figure}\label{fig:simple-gauss}
    \centering
\begin{tikzpicture}
    \node [circle,draw=black,fill=lightgray](1) {$S_0$};
    \node [circle,draw=black,fill=lightgray,right =of 1] (2) {$S_1$};
    \node [circle,draw=black,fill=lightgray,right =of 2] (3) {$S_2$};
    \node [right =of 3] (4) {$\dots$};
    \node [circle,draw=black,fill=lightgray,right =of 4] (5) {$S_t$};
    
    \draw[Arrow] (1.east) -- (2.west);
    \draw[Arrow] (2.east) -- (3.west);
    \draw[Arrow] (3.east) -- (4.west);
    \draw[Arrow] (4.east) -- (5.west);

    % \node [circle,draw=black,fill=lightgray,below =of 2] (6) {$\Sigma(S_1)$};
    % \node [circle,draw=black,fill=lightgray,below =of 3] (7) {$\Sigma(S_2)$};
    % \node [circle,draw=black,fill=lightgray,below =of 5] (8) {$\Sigma(S_t)$};

    % \node [circle,draw=black,fill=lightgray,below =of 6] (9) {$\mu(S_1)$};
    % \node [circle,draw=black,fill=lightgray,below =of 7] (10) {$\mu(S_2)$};
    % \node [circle,draw=black,fill=lightgray,below =of 8] (11) {$\mu(S_t)$};
    
    \node [circle,draw=black,below =of 2] (12) {$R_1$};
    \node [circle,draw=black,below =of 3] (13) {$R_2$};
    \node [circle,draw=black,below =of 5] (14) {$R_t$};
    
    % \node [circle,draw=black,below =of 2] (10) {$R_1$};
    % \node [circle,draw=black,below =of 3] (11) {$R_2$};
    % \node [circle,draw=black,below =of 5] (12) {$R_t$};
    
    \draw[Arrow] (2.south) -- (12.north);
    \draw[Arrow] (3.south) -- (13.north);
    \draw[Arrow] (5.south) -- (14.north);

    % \draw[Arrow] (6.south) -- (9.north);
    % \draw[Arrow] (7.south) -- (10.north);
    % \draw[Arrow] (8.south) -- (11.north);

    % \draw[Arrow] (9.south) -- (12.north);
    % \draw[Arrow] (10.south) -- (13.north);
    % \draw[Arrow] (11.south) -- (14.north);

    % \draw[Arrow] (2) to [out=25, in=160] (4); 
\end{tikzpicture}
\caption{
    Depiction of the underlying economic process. $S_t$ is a Markov state process drawn from $S_t \sim P(S_t, S_{t-1})$, while $R_t$ is a multivariate Gaussian of dimension $N$ (one for each firm). $R_t$ is Gaussian only conditional on $S_t$, i.e. $R_t \mid S_t \sim \mathcal{N}(\mu(S_t), \Sigma(S_t))$.
}
\end{figure}

Investors do not observe all elements of $R_t$ simultaneously. Rather, they observe them sequentially --- denote a partition of $R_t$ after $n \le N$ firms have been observed with $R_{t,1:n}$. Write the distributions of observed returns $R_{t,A}$ and the returns yet to be observed $R_{t,B}$ as the partitions

\begin{align*}
    R_t &= \begin{bmatrix}
        R^A_t & R^B_t
    \end{bmatrix}' \\ 
    R^A_{t} &\sim \mathcal{N}(\mu^A, \Sigma^A) \\
    R^B_{t} &\sim \mathcal{N}(\mu^B, \Sigma^B) \\
\end{align*}

\noindent where 

$$
E[R_t \mid S_t] = \mu = \begin{bmatrix}
    \mu_A \\
    \mu_B
\end{bmatrix}
$$

The superscripts $A$ and $B$ are used to indicate a subsetting operation where $\mu^A = \mu_{1:n}$ and $\mu_B = \mu_{n+1:N}$. The covariance matrices can be similarly partitioned into a block matrix

$$
\Var[R_t \mid S_t] = \underset{N \times N}{\Sigma} = \begin{bmatrix}
    \underset{n\times n}{\Sigma^A} & \underset{n\times N-n}{\Sigma^{AB}} \\
    \underset{N-n\times n}{\Sigma^{BA}} & \underset{N-n\times N-n}{\Sigma^B} \\
\end{bmatrix}
$$

Note, however, that $R_t^A$ is observed, and is no longer a stochastic variable. However, it was \textit{drawn} from a distribution correlated to $R_t^B$, and thus can be used as conditioning information to more precisely determine the distribution of the returns to be revealed, $R_t^B$. It can be shown \todo{Actually show this?} that, conditional on observing $R_t^A$, the distribution of $R_t^B$ is

$$
R_t^B \mid R_t^A, S_t \sim \mathcal{N} (\overline \mu^B, \overline \Sigma^B)
$$

\noindent for conditional parameters

\begin{align*}
    \overline \mu^B &= \mu^B + \Sigma^{BA} (\Sigma^{A})^{-1}(R_t^A - \mu^A)\\
    \overline \Sigma^B &= \Sigma^B + \Sigma^{BA} (\Sigma^{A})^{-1}\Sigma^{AB}
\end{align*}

\newcommand{\Shape}{\mathbf{\Omega}}

I assume that, for any state, the covariance matrix of firm payoffs is drawn from an inverse Wishart distribution parameterized by the shape matrix $\Shape$ and precision $\nu$. The distribution of $\Sigma$ holds even for cases where the covariance matrix does not change across states\footnote{In models where the covariance matrix changes with state (cases 3 and 4), the inverse Whishart distribution can still be used as the distribution collapses with certainty as $\nu \rightarrow \infty$.}. The matrix $\Shape$ determines the fundamental "shape" of the covariance structure, in that the mean of the distribution of $\Sigma$ is 

$$
E[\Sigma \mid S_t] = \frac{\Shape}{\nu - N - 1}
$$

The average \textit{covariance} can vary substantially in terms of scale as $\nu$ changes, but the average \textit{correlation} remains the same regardless of $\nu$. Any two draws $\Sigma_1$ and $\Sigma_2$ can have highly varied behavior. For example, $\Sigma_1$ might suggest a negative correlation in the payoffs of two firms, while $\Sigma_2$ could suggest a positive correlation. The inverse Wishart distribution is advantageous because the covariance matrix governing firm payoffs can vary meaningfully between states, and the distribution's properties are well-known.



\end{document}