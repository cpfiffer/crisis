\documentclass{article}

\title{Crisis Learning}
\author{Cameron Pfiffer}

\usepackage{amsmath}
\usepackage{palatino}
\usepackage{tikz}
\usepackage{amssymb}

\usetikzlibrary{positioning, calc, 
shapes.geometric, shapes, 
shapes.multipart, arrows.meta, 
arrows, decorations.markings, 
external, trees}

\begin{document}

\tikzstyle{Arrow} = [
	thick, 
	decoration={
		markings,
		mark=at position 1 with {
			\arrow[thick]{latex}
			}
		}, 
	shorten >= 3pt, preaction = {decorate}
	]

\maketitle

Howdy hey, gang. The goal of this document is to describe the rough shape of an economy with a simple state-based structure. 

\section{Model styles}

In all models, it is assumed that the initial state distribution is categorically distributed with probabilities $\Pi = [\pi_1, \pi_2]$. We'll keep it simple for now and use only two possible states, but in general everything should function with an arbitrary number of states. 

\subsection{Simple Gaussian}

Figure \ref{fig:simple-gauss} demonstrates an extremely simple economy where a latent state variable $S_t$ evolves over time. In each possible state $S_t \in \mathbb{S}$, the location and scale parameters ($\mu(S_t)$ and $\Sigma(S_t)$) of the Gaussian return vector $R_t$ may vary:

\begin{enumerate}
    \item The degenerate constant mean/variance condition
    $\Sigma(S_t)= \Sigma$ and a changing mean return vector, $\mu(S_t) = \mu$.
    \item A constant variance term $\Sigma(S_t) = \Sigma(S_t') = \Sigma$ and a changing mean return vector, $\mu(S_t) \ne \mu(S_t')$.
    \item A varying variance term $\Sigma(S_t) \ne \Sigma(S_t')$ and a constant mean return vector, $\mu(S_t) = \mu$.
    \item The "everything changes" case where both $\mu(S_t)$ and $\Sigma(S_t)$ vary.
\end{enumerate}

Investors do not observe all elements of $R_t$ simultaneously. Rather, they observe them sequentially -- 

\begin{figure}\label{fig:simple-gauss}
    \centering
\begin{tikzpicture}
    \node [circle,draw=black,fill=lightgray](1) {$S_0$};
    \node [circle,draw=black,fill=lightgray,right =of 1] (2) {$S_1$};
    \node [circle,draw=black,fill=lightgray,right =of 2] (3) {$S_2$};
    \node [right =of 3] (4) {$\dots$};
    \node [circle,draw=black,fill=lightgray,right =of 4] (5) {$S_t$};
    
    \draw[Arrow] (1.east) -- (2.west);
    \draw[Arrow] (2.east) -- (3.west);
    \draw[Arrow] (3.east) -- (4.west);
    \draw[Arrow] (4.east) -- (5.west);

    \node [circle,draw=black,below =of 2] (6) {$R_1$};
    \node [circle,draw=black,below =of 3] (7) {$R_2$};
    \node [circle,draw=black,below =of 5] (9) {$R_t$};
    
    \draw[Arrow] (2.south) -- (6.north);
    \draw[Arrow] (3.south) -- (7.north);
    \draw[Arrow] (5.south) -- (9.north);

    % \draw[Arrow] (2) to [out=25, in=160] (4); 
\end{tikzpicture}
\caption{
    Depiction of the underlying economic process. $S_t$ is a Markov state process drawn from $S_t \sim P(S_t, S_{t-1})$, while $R_t$ is a multivariate Gaussian of dimension $K$ (one for each firm). $R_t$ is Gaussian only conditional on $S_t$, i.e. $R_t \mid S_t \sim N(\mu(S_t), \Sigma(S_t))$.
}
\end{figure}

\end{document}